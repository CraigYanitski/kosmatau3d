\documentclass[a4paper]{article}
\usepackage[T1]{fontenc}
\usepackage[utf8]{inputenc}
\usepackage[top=2.5cm,left=3cm,right=3cm,bottom=2.5cm,dvips]{geometry}

\newcommand{\kosmatau}{\texttt{KOSMA-\(\tau\)}}
\newcommand{\kttd}{\texttt{kosmatau3d}}

\begin{document}
	
    \title{The interaction of \texttt{kosmatau3d} with ISMDB}
    \author{Craig Yanitski}
    \date{}
    \maketitle

    The three-dimensional PDR modelling code \kttd, based on \kosmatau, will need to interact with ISMDB\footnote{https://ismdb.obspm.fr/} in order to download the required data.
    From what I can find thus-far, the functionality of ISMDB is to give the appropriate parameters to download the data for each model individually.
    (ex. For an isochoric model with \(\chi=10\), \(n_H=35\), and \(A_V=10\) the model ID \texttt{P154G3\_n\_r1e1n3p5e1A1e1}, and the downloadable data has an additional suffix depending on the data type.)
    This allows one to examine the profile of various properties such as density, abundance, charge grain, etc.
    Technically this is also possible for \kosmatau.
    I have written python code using \texttt{requests} and \texttt{beautifulsoup} to determine the available grids and access the data, but it is far from efficient.%

	\hfill \\

    What I need to make {\kttd} work is a complete grid with results from multiple models (Currently I have a parameter space spanning 1225 models with a few holes).
    Due to the spherical structure of the {\kosmatau} model and how it is implemented in \kttd, I will require simple clump-averaged surface values.
    The ideal method to perform such a task is to query the database specifying the input parameter space and type of output data.
    It is not downloading each model's data individually, but to send some SQL query to ISMDB and retrieving a data file (for example by using \texttt{requests} in python).
    This is normally handled in python using \texttt{astroquery}.
    This will not work for a new database, but its functionality can be inherited by whatever function I write.
    The main advantage is that I can query the database with minimal 'brute force' attempts to determine the database structure and available parameter space.
    The primary issue is whether this is possible in ISMDB.%
    
    \hfill \\

    So far the type of queries that I will send to the database will have the logspace parameters metalicity, surface hydrogen density, hydrogen mass, far-UV radiation, and primary cosmic ray ionisation rate as input parameters, and output will be intensity, optical depth, and far-UV extinction.
    Each row in the data output will have a column for each input parameter, followed by the output data.
    For the getting the \(A_{FUV}\), this is independent of the far-UV radiation and cosmic ray ionisation rate and has just one output value.
    The intensities and optical depths, however, will contain all of the input parameters and have an output for each transition\footnote{Currently I have 171 transitions, but this depends on the model.} in the model.
    This also applies to the values at all of the 333 dust wavelengths.
    In the end, the current grid files in {\kttd} range in size from 3 kB to 12 MB.%
    
    \hfill \\
    
    Ideally I should be able to query ISMDB specifying the parameter space and desired transitions, and it should be able to return a plaintext table of the data.
    In principle this should not be a static parameter space that will be queried, but I should be able to specify the input and output parameters in the SQL query.
    For example it will be useful to use the number densities or temperature as input parameters or output data.
    There should also be a way to ask the database what parameters are available.%
    
\end{document}