\documentclass[a4paper]{article}
\usepackage{graphicx, hyperref, amsmath, amsfonts, hhline, booktabs}
%\usepackage[fleqn,tbtags]{amsmath}
\usepackage[flushleft]{threeparttable}
\usepackage{mathtools}
\usepackage[T1]{fontenc}
\usepackage[utf8]{inputenc}
%\usepackage{amsmath,amssymb,array,bm,booktabs,color,float,graphicx,hhline,hyperref,indentfirst,longtable,morefloats,natbib,parskip,pdflscape,setspace,subcaption,titlesec,wrapfig,xtab}
%\usepackage[top=2.5cm,left=2.5cm,right=2.5cm,bottom=2.5cm,dvips]{geometry}
%\usepackage[flushleft]{threeparttable}
% \bibliographystyle{abbrvnat}
%\setcitestyle{authoryear,open={(},close={)}}
%\bibliographystyle{aa}
%\hypersetup{}

%\setcounter{tocdepth}{2}

%\setlength{\parindent}{2em}
%\setlength{\parskip}{0em}

%\titleformat*{\subsubsection}{\large\itshape}

\newcommand{\mathsc}[1]{{\normalfont\textsc{#1}}}
\newcommand{\kosmatau}{KOSMA-\(\tau\)}

\begin{document}
    \title{Single-voxel debugging}
    \author{Craig Yanitski}
    \maketitle

    \section{Single-voxel setup}
    \label{setup}

    Here we will examine the current setup of the calculations of \kosmatau3d.
    These will give a basis on how the probabilistic calculation works for a single voxel, and what assumptions are present when using the single-voxel model.
    The observed intensity is calculated from the radiative transfer equation,

    \[
    dI_\nu = - I_\nu \kappa_\nu ds + \epsilon_nu ds,
    \]

    we can assume \(\kappa\) and \(\epsilon\) (the absorption and emissivity coefficients, respectively) are constant in the voxel.
    Therefore they are calculated using the voxel-averaged emmision:

    \[
    \epsilon_\nu (v_{obs}) = \frac{\left<I_\nu\right> (v_{obs})}{\Delta s},
    \]

    \[
    \kappa_\nu (v_{obs}) = \frac{\left<\tau_\nu\right> (v_{obs})}{\Delta s},
    \]

    In order to properly understand these voxel-averaged values, it is necessary to have some insight for what is happening inside the voxel.
    The clumpy PDR is approximated by an ensemble of spherically-symetric \kosmatau 'clumps'.
    The ensemble of clumps has a Gaussian velocity distribution with dispersion \(\sigma_{ens}\), and each clump has an intrinsic velocity dispersion \(\sigma_{cl}\) (this value depends on which underlying \kosmatau grid is used, but currently it is \(\sim0.71 \frac{km}{s}\)).
    The number of clumps with a particular mass (\(N_j\)) follows the mass spectrum:

    \[
    \frac{\mathrm{d}N}{\mathrm{d}M} = A M^{-\alpha}
    \]

    Here we adopt \(\alpha=1.84\) from Heithausen et al. (1998), and \(A\) is a constant.
    Henceforth the subscript \(j\) will refer to a property specific for clumps with mass \(M_j\).
    Clumps can be observed with a particular radial velocity, so we also use the formalism of the subscript \(i\) to refer to a property with radial velocity \(v_i\).
    Note that this is a velocity within the voxel, not the observed velocity.
    The observed radial velocity is \(v_{obs}\).

    Now the number of clumps with a particular radial velocity \(v_i\) is determined by the number of clumps of a particular mass \(N_j\) and velocity distribution of the ensemble \(\sigma_{ens}\).

    \[
    \Delta N_{j,i} (v_{obs}) = \frac{N_j}{\sqrt{2 \pi} \sigma_{ens}} \ \mathrm{exp} \left( -\frac{(v_{vox}-v_i)^2}{2 \sigma_{ens}} \right) \Delta v
    \]

    \(p_{j,i}\) is the probability of having \(k_{j,i}\) clumps in a line-of-sight observed at velocity \(v_i\), determined by Binomial distribution of the number of clumps at that velocity:

    \[
    p_{j,i} = \binom{N_{j,i}}{k_{j,i}} p_j^{k_{j,i}} (1-p_j)^{N_{j,i}-k_{j,i}}
    \]

    Where \(p_j\) is the area fraction of clump \(j\).
    If \(p_j N_{j,i}>5\), then a gaussian expression is used with \(\mu_{j,i} = p_j N_{j,i}\) and \(\sigma_{j,i} = \sqrt{N_{j,i} p_j (1-p_j)}\).

    For having \(k_{j,i}\) clumps in a line-of-sight, the intensity and optical depth (\(I_{\nu,j,i}\) \& \(\tau_{\nu,j,i}\)) can be calculated.
    Here the subscript \(\nu\) means the properties refer to a particular transition (which is what we get from the \kosmatau grid).

    \[
    I_{\nu,j,i} (v_{obs}) = k_{j,i} I_{cl,j} \mathrm{exp} \left( \frac{(v_i-v_{obs})^2}{2\sigma_{cl}^2} \right)
    \]

    \[
    \tau_{\nu,j,i} (v_{obs}) = k_{j,i} \tau_{cl,j} \mathrm{exp} \left( \frac{(v_i-v_{obs})^2}{2\sigma_{cl}^2} \right)
    \]

    Finally the intensity and optical depth can be averaged over the voxel using all possible combinations and their corresponding probabilities to determine voxel properties we require.

    \[
    \left< I_\nu \right> (v_{obs}) = \sum_i \left( \prod_j p_{j,i}(k_{j,i}) \right) \left( \sum_j k_{j,i} I_{\nu,j,i} (v_{obs}) \right)
    \]

    \[
    \left< \tau_\nu \right> (v_{obs}) = -\mathrm{ln} \left[ \sum_i \left( \prod_j p_{j,i}(k_{j,i}) \right) \ \mathrm{exp} \left( - \sum_j k_{j,i} \tau_{\nu,j,i} (v_{obs}) \right) \right]
    \]

    Here the sums and products ensure we consider contributions from all clump masses at all radial velocities, since there is a velocity dispersion to each clump's intensity and optical depth.
    Now we can finally substitute these into the radiative transfer equation to determine the intensity we expect to observe from this voxel.

    \[
    I_{\mathrm{vox},\nu} (v_{obs}) = \frac{\epsilon_\nu (v_{obs})}{\kappa_\nu (v_{obs})} \left( 1 - e^{-\kappa_\nu (v_{obs}) \Delta s} \right) + I_{bg} e^{-\kappa_\nu (v_{obs}) \Delta s}
    \]

    \section{Calculation error}

    There is clearly some issue with the single voxel calculation, since there is an absorption feature that should not be present.
    Perhaps this was noticed initially when setting up the single-voxel model, but at the time it was dismissed due to the nonphysical model parameters (a volume filling factor \(f_V > 1\)).
    Yoko has explored more of the parameter space to compare with observations, though, and encountered this problem again.
    In this use-case, an over-filled voxel is used to compare to various lines-of-sight.
    This makes enough sense as the clumps overflow behind the voxel.
    I now need to revise my intensity calculations not only to correct this absorption feature, but to ensure that the previous calculations (without this error) also agree.

    For a densely-filled voxel, there should exist a plateau in the \(I(v_{obs})\) curve (see Figure \ref{plateau}), however the error is that there still exists some additional absorption.
    Due to the calculation of the voxel intensity (see \S \ref{setup}), there are two separate modes of calculation: standard and normalised.
    The standard calculation of the intensity uses the clump numbers calculated using \(\Delta N_{j,i}\), while the normalised calculation resizes the voxel to ensure a given number of the largest clump are included in the voxel (typically 1 for the dense clumpy ISM).
    There is also part of the code which resizes the voxel if it is too small to fit the largest clump.
    It is the combination of these two features that make debugging this error more difficult.
    For that reason, I will debug these calculation methods separately.

    \subsection{The correction}

    The issue identified thus-far is that the probabilistic description has been applied to the averaged \textit{intensity} rather than the \textit{emissivity}, and therefore there exists some excess absorption in the plateau region.
    The proposed fix for this is to modify the equation for \(I_{\nu, j, i} (v_{obs})\) to calculate instead \(\epsilon_{\nu, j, i} (v_{obs})\):

    \[
    \epsilon_{\nu,j,i} = I_{\nu,j,i} \frac{\tau_{\nu,j,i}} {L_{j}} / [ 1 - \exp(-\tau_{\nu,j,i}) ],
    \]

    leading to,

    \[
    \left< \epsilon_\nu \right> (v_{obs}) = \sum_i \left( \prod_j p_{j,i}(k_{j,i}) \right) \left( \sum_j k_{j,i} \epsilon_{\nu,j,i} (v_{obs}) \right).
    \]

    The modification for the optical depth in terms of opacity is,

    \[
    \left< \kappa_\nu \right> (v_{obs}) = -\mathrm{ln} \left[ \sum_i \left( \prod_j p_{j,i}(k_{j,i}) \right) \ \mathrm{exp} \left( - \sum_j \frac{k_{j,i} \tau_{\nu,j,i} (v_{obs}) \times A_{j}}{V_{voxel}} \right) \right]
    \]

    At least this is how the intended modification to the code should work.
    Its implementation tries to remain as faithful to this description as possible, but it was noted that there are still a couple errors.
    Therefore some flags were added to choose the type of implementation: \texttt{test\_calc}, \texttt{test\_pexp}, and \texttt{test\_opacity}.
    \texttt{test\_calc} chooses whether or not the test calculation is used, and it will definitely affect the emissivity.
    \texttt{test\_opacity} is a flag to use either the original definition of opacity or to use the modified version given above.
    Finally, \texttt{test\_pexp} was a test to see how the calculation of opacity changes if one considers how the result is changed if one applies the probabilistic approach to \(\kappa\) rather than \(e^{-\kappa}\).
    This final one should not be working since the probabilistic approach should work on the attenuation (\(e^{-\tau}\)).
    In any case, these are the parameters that will be varied in the following investigation.

    \subsection{Standard calculation}



\end{document}